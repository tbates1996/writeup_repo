\title{CS161 Project 3 Writeup}
\author{
        Thomas Bates \\
				Jerry Evarts \\
        CS161 Fall 2018\\
        UC Berkeley \\
}
\date{\today}

\documentclass[12pt]{article}

\begin{document}
\maketitle

\section{Weaponizing the Vunerability}

\section{Vulnerability Writeup}
\subsection{Hidden Field Profile Page}
When we navigate to the profile page we are given a form to update the users avatar and age. However, there is also an input that is hidden on the page. The way this is done in the application is using CSS to make the input hidden, but this doesnt stop attackers from modifying the input on their developer console and removing the style which would display the textbox. The original HTML is displayed bellow \\
$<$ input type="text" class="input" style="display:none" id="username" name="username" value="{{ username }}"$>$ \\
In order to fix this problem we could validate the that username in the form is the username of the person who is currently using the session. We could also completely remove the field and just use the username that is stored with the session and process the request using that username. There is no reason to post the username with the form if we can access it with the session cookie.
\subsection{}
\section{Other Issues}

\section{Mandatory Feedback}
I think that the project as a whole was great; however, it can be improved by teaching us a little more about how to preform exploits outside of what is covered in class. I feel that we learn about the vunerabilies, but it would be nice to see more examples of how it's done in practice. Our favorite part of the project was using the SESSION\_ID cookies to do a SQL Injection to gain access to dirks account. It helped show how dangerous SQL injection attacks could be. In addition to that I was interesting to see how you could take this information and try to find exploits on actual website without having the source code.
\end{document}
