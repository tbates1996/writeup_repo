\title{CS161 Project 3 Writeup}
\author{
        Thomas Bates \\
				Jerry Evarts \\
        CS161 Fall 2018\\
        UC Berkeley \\
}
\date{\today}

\documentclass[12pt]{article}

\begin{document}
\maketitle

\section{Weaponizing the Vunerability}
Opening the developer console on any logged in account for Snapitterbook, we can perform a SQL injection using the document.cookie variable.  We know there exists a user named dirks, because there exists two existing posts on the site from dirks.  Because Session ID is used to verify which user is present, we can exploit this by injecting sql into the developer console that spoofs the current username.  This allows us to trick the browser into thinking we are the user dirks.  We now have the ability to post on dirks's wall which any user can visit from the home page.  Using this, we can post a script on dirks's wall that is invisible to the user and performs a request to a malicious website by updating the document.location variable to a maliciouwith the document.cookie appended to the request.  The malicious site would receive the cookie, store it, and then redirect the user back to the original page they were on to mask the fact that they were redirected.  This would all occur immediately and the user would not be aware.

use document.cookie as parameter
\section{Vulnerability Writeup}
\subsection{Hidden Field Profile Page}
When we navigate to the profile page we are given a form to update the users avatar and age. However, there is also an input that is hidden on the page. The way this is done in the application is using CSS to make the input hidden, but this doesnt stop attackers from modifying the input on their developer console and removing the style which would display the textbox. The original HTML is displayed bellow \\
$<$ input type="text" class="input" style="display:none" id="username" name="username" value="{{ username }}"$>$ \\
In order to fix this problem we could validate the that username in the form is the username of the person who is currently using the session. We could also completely remove the field and just use the username that is stored with the session and process the request using that username. There is no reason to post the username with the form if we can access it with the session cookie.
\subsection{Wall Other User XSS}
In the path for localhost:5000/wall/$<$other-user$>$ the server code does not escape HTML for the username, so it is vunerable to a XSS attack. This is an issue because if you enter a img tag with a bad src url and an onerror tag with malicious javascript then it will be posted in the no\_wall.html page and run the script in the onerror tag. This could be used as a reflected XSS attack where an attacker could post a link to a wall with one of these img scripts as the other-username and then steal a users cookies. When the server checks to see that the user exists and it fails it will then render the no\_wall page using: \\
	return render\_template('no\_wall.html', username=other\_username) \\
Then the no\_wall.html will render this user name in the HTML tag:\\
$<$ h1$>$ class="title"  No wall for \{\{ username \}\}! $<$ \/h1 $>$\\
This will render the $<$img src="sfddfs" onerror="alert()" $>$ on the page allowing the attacker to run any code they want. In order to fix this exploit we should escape the HTML of the other-username that gets passes into the URL. This will prevent the attacker from being able to create this malicious URL that can be then used as a reflected XSS.
\section{Other Issues}
We can create a malicious site which has a form that the user could fill out which has a hidden post request back to Snapitterbook.  By observing the structure of post requests on snapitterbook, an adversary could exploit the structure in their own post request.  The form on the malicious site would disable the referrer header so that it avoids the CSRF detection in server.py.  Observing the server.py file, on line 44 we request the referer header in order to protect all of our functions against CSRF attack.  This could be protected authentication tokens that are sent alongside the post.



\section{Mandatory Feedback}
I think that the project as a whole was great; however, it can be improved by teaching us a little more about how to preform exploits outside of what is covered in class. I feel that we learn about the vunerabilies, but it would be nice to see more examples of how it's done in practice. Our favorite part of the project was using the SESSION\_ID cookies to do a SQL Injection to gain access to dirks account. It helped show how dangerous SQL injection attacks could be. In addition to that I was interesting to see how you could take this information and try to find exploits on actual website without having the source code.
\end{document}
